\documentclass{article}

\usepackage{amsmath, amsfonts, amssymb, amsthm}
\usepackage{xcolor}

\usepackage{hyperref}

% \hypersetup{
%     colorlinks=true,
%     linkcolor=blue,
%     filecolor=magenta,      
%     urlcolor=cyan,
% }

\renewcommand{\qedsymbol}{\rule{0.7em}{0.7em}}

\newcommand{\floor}[1]{\left\lfloor #1 \right\rfloor}
\newcommand{\ceil}[1]{\left\lceil #1 \right\rceil}

\def \d[#1]#2 {\frac{\mathrm{d}^{#2}}{\mathrm{d} #1^{#2}}}
\def \del[#1]#2 {\frac{\partial^{#2}}{\partial #1^{#2}}}



\usepackage{graphicx}

\graphicspath{{.\\numerisk_øving\\bilder\\}}

\title{Numerical exerice in classical mechanics}
\author{Martin Johnsrud}


\begin{document}
    \maketitle

    \section*{Introduction}
    This text explores several different methods for solving differential equations to model a pendulum. The methods are anlyzed for different time steps, to see the advatages of running the different methods.
    
    \section*{Theory}
    Using the small angle approximation for a pendulum, we get the differential equation
    
    \begin{equation*}
        \d[t]2 \theta(t) = -\frac{g}{l} \theta(t)
    \end{equation*}
    This can be rewritten in the form
        \begin{equation*}
            \dot y = \d[t]
            \begin{pmatrix}
                \theta \\
                \dot \theta
            \end{pmatrix}
            = f 
            \begin{pmatrix}
                \theta \\
                \dot \theta
            \end{pmatrix}
             =
            \begin{pmatrix}
                \dot \theta \\
                -\frac{g}{l} \theta,
            \end{pmatrix} 
        \end{equation*}  
    making explicit differential equation solvers straight forward. Eulers method is
    \begin{equation*}
        y_{n + 1} = f(y_n) \Delta t,
    \end{equation*}
    and Runge Kutta 4 is
    \begin{align*}
        k_1 &= f(y_n) ,\\
        k_2 &= f(y_n + k_1 / 2)\Delta t, \\
        k_3 &= f(y_n + k_2 / 2)\Delta t, \\
        k_4 &= f(y_n + k_3)\Delta t, \\
        y_{n + 1} &= y_n + \frac{1}{6}(k_1 + 2k_2 + 2k_3 + k_4).
    \end{align*}

    As we will see later, these methods does not conserve energy. This is achived with the implicit Eurler-Cromer method

    \begin{align*}
        \dot \theta_{n + 1} = \dot \theta_{n} - \frac{g}{l} \theta \\
        \theta_{n + 1} = \dot \theta_{n + 1} \Delta t
    \end{align*}

\section*{Results}
    \subsection*{Eulers method}

        \begin{figure}[h]
            \includegraphics[width = \textwidth]{{euler_0.pdf}}
            \caption{The upper plot shows the oscillations of the pendulum with different time-steps, comparing them to the analytical solution. The lower plot shows how the energy of the different systems increases with time.}
            \label{Eulers method}
        \end{figure}        

        Figure \ref{Eulers method} shows the resultus of moddeling the system with Eurlers method for using different time steps, $\Delta t \in \{0.005, 0.02, 0.045\}$, over a time period $t \in [0, 10]$. Based on this analysis, we can see that a time step of around $\Delta t = 0.005$ is sufficent for accurate results, however for large enough time periods, all timesteps makes the method unstable.

    \subsection*{Comparison with other methods}

        Figure \ref{Phase space} and \ref{All methods} compares Eulers method with Euler-Cromer and Runge-Kutta 4 using the same time step ($\Delta t = 0.5$). This comparison shows that Eulers method is much less stabel than the other methods. Runge-Kutta 4 is more computationally expensive as it is a method of a higher order, however Euler-Cromer achives much higher stability without being more expensive.

        \begin{figure}[h]
            \hspace{-0.1 \textwidth}
            \includegraphics[width = 1.2\textwidth]{{all_methods_phase.pdf}}
            \caption{This figure shows motion of the pendulum through phase space, the space of coordinates $(\theta, \dot \theta)$, parametrized by time $t$. This is a plot of the same motoion as in figure \ref{All methods}.}
            \label{Phase space}
        \end{figure}  

        \begin{figure}[h]
            \includegraphics[width = \textwidth]{{all_methods_0.pdf}}
            \includegraphics[width = \textwidth]{{all_methods_1.pdf}}
            \includegraphics[width = \textwidth]{{all_methods_2.pdf}}
            \caption{A comparison between the methods discussed in the text. Time is plotted against the angle $\theta$ of the pendulum at the left, and the energy $E$ at the right. From top to bottom, the methods are Euler, Euler-Cromer and Runge-Kutta 4. }
            \label{All methods}
        \end{figure}  

    \subsection*{Euler-Cromer vs. Runge-Kutta 4}
        
        \begin{figure}[h]
            \includegraphics[width = \textwidth]{{ECvsRK4_phase.pdf}}
            \caption{The path in phase space of Euler-Cromer method and Runge-Kutta 4, corresponding to the movements in figure \ref{EC vs RK4}}
            \label{Phase space 2}
        \end{figure}
        
        \begin{figure}[h]
            \includegraphics[width = \textwidth]{{ECvsRK4_0.pdf}}
            \includegraphics[width = \textwidth]{{ECvsRK4_1.pdf}}
            \caption{Comparison of the Euler-Cromer method and Runge-Kutta 4.}
            \label{EC vs RK4}
        \end{figure}
        
        Though both Euler-Cromer and Runge-Kutta 4 are more stable than Euler, they have different advantages. Euler-Cormer conserves energy, so however long the simulation is run for, it will not diverge. The disadvatage of Euler-Cromer can be seen in the phase diagram in figure \ref{Phase space 2}. While the Runge-Kutta path keeps in a (jagged) circle, the path of the Euler-Cromer is elongated when the timestep increases. This means that even though Euler-Cromer is more stabel, it deviates more in the start of the simulation. However, as we can see from figure \ref{EC vs RK4}, Runge-Kutta 4 looses energy. It will therefore always diverge further from the analytical solution the longer the simulation is run for. For the interval tested here ($t = 10$), Euler-Cromer is fairly accurate with $\Delta t = 0.04$, while Runge-Kutta 4 remains accurate enough up to $\Delta t = 0.16$. For longer simulations, however, Runge-Kutta would need shorter steps.    

\end{document} 