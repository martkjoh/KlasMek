\documentclass{article}

\usepackage{amsmath, amsfonts, amssymb, amsthm}
\usepackage{xcolor}

\usepackage{hyperref}

% \hypersetup{
%     colorlinks=true,
%     linkcolor=blue,
%     filecolor=magenta,      
%     urlcolor=cyan,
% }

\renewcommand{\qedsymbol}{\rule{0.7em}{0.7em}}

\newcommand{\floor}[1]{\left\lfloor #1 \right\rfloor}
\newcommand{\ceil}[1]{\left\lceil #1 \right\rceil}

\def \d[#1]#2 {\frac{\mathrm{d}^{#2}}{\mathrm{d} #1^{#2}}}
\def \del[#1]#2 {\frac{\partial^{#2}}{\partial #1^{#2}}}



\title{Numerical exerice in classical mechanics}


\begin{document}
    \section*{Introduction}
    This text explores several different methods for solving differential equations to model a pendulum. The methods are anlyzed for different time steps, to see the different expences of running the methods.
    
    \section*{Theory}
    Using the small angle approximation for a pendulum, we get the differential equation
    
    \begin{equation*}
        \d[t]2 \theta(t) = -\frac{g}{l} \theta(t)
    \end{equation*}
    This can be rewritten in the form
        \begin{equation*}
            \dot y = \d[t]
            \begin{pmatrix}
                \theta \\
                \dot \theta
            \end{pmatrix}
            = f 
            \begin{pmatrix}
                \theta \\
                \dot \theta
            \end{pmatrix}
             =
            \begin{pmatrix}
                \dot \theta \\
                -\frac{g}{l} \theta,
            \end{pmatrix} 
        \end{equation*}  
    making explicit differential equation solvers straight forward. Eulers method is
    \begin{equation*}
        y_{n + 1} = f(y_n) \Delta t,
    \end{equation*}
    and Runge Kutta 4 is.
    \begin{align*}
        k_1 &= f(y_n) ,\\
        k_2 &= f(y_n + k_1 / 2)\Delta t, \\
        k_3 &= f(y_n + k_2 / 2)\Delta t, \\
        k_4 &= f(y_n + k_3)\Delta t, \\
        y_{n + 1} &= y_n + \frac{1}{6}(k_1 + 2k_2 + 2k_3 + k_4).
    \end{align*}
\end{document}