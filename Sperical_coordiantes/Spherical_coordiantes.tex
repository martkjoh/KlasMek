\documentclass{article}

\usepackage{amsmath}

\title{Deriving the laplacian in spherical cordinates}
\author{Martin Johnsrud}

\begin{document}
    \maketitle
    The definition of spherical coordinates is
    \begin{align*}
        x = r \cos(\phi)\sin(\theta) \quad & r = \sqrt{x^2 + y^2  + z^2}, \quad r \in [0, \infty) \\
        y = r \sin(\phi)\sin(\theta) \quad & \phi = \arctan{\bigg( \frac{y}{x} \bigg)}, \quad \phi \in [0, 2 \pi)\\
        z = r \cos(\theta) \quad & \theta = \arctan{\bigg( \frac{\sqrt{x^2 + y^2}}{z} \bigg)}, \quad \theta \in [0, \pi)
    \end{align*} 
    To see the inverse relationship, we just substitute
    %
    \begin{align*}
        r = &\sqrt{(r \cos(\phi)\sin(\theta))^2 + (r \sin(\phi)\sin(\theta))^2  + (r \cos(\theta) )^2} \\ 
        = &r \sqrt{(\cos^2{\phi} + \sin^2{\phi})\sin^2{\theta} + \cos^2{\theta}} = r \\ 
        \phi = &\arctan{\bigg( \frac{r \sin(\phi)\sin(\theta)}{r \cos(\phi)\sin(\theta)} \bigg)} = \arctan{(\tan{\phi})} = \phi \\
        \theta = & \arctan{\bigg( \frac{\sqrt{(r \cos(\phi)\sin(\theta) )^2 + (r \sin(\phi)\sin(\theta))^2}}{r \cos(\theta)} \bigg)} \\ =& \arctan{\bigg( \frac{\sqrt{\sin^2(\theta)}}{\cos(\theta)} \bigg)} = \theta.
    \end{align*}
    %
    The differential of a function can be written as
    %
    \begin{align*}
        \mathrm{d}f = \nabla f \cdot \mathrm{d}\vec r,    
    \end{align*}
    %
    where \(\mathrm{d} \vec r\) is a differential line element. In cartesian cooridnates, this simply becomes
    \begin{equation*}
        \mathrm{d}f = \frac{\partial f}{\partial x} \mathrm{d}x + \frac{\partial f}{\partial y} \mathrm{d}y + \frac{\partial f}{\partial z} \mathrm{d}z
    \end{equation*}
    %
    In spherical coordinates, a differential line element is 
    %
    \begin{equation*}
        \mathrm{d} \vec r = \hat r \mathrm{d}r + \hat \phi r \sin(\theta) \mathrm{d} \phi + \hat \theta \mathrm{d} \theta,
    \end{equation*}
    %
    %
    To find nabla represented in spherical coordinates, we simply compare the terms in
    %
    \begin{align*}
        \nabla f \cdot (\hat r \mathrm{d}r + \hat \phi r \sin(\theta) \mathrm{d} \phi + \hat \theta \mathrm{d} \theta) = & \frac{\partial f}{\partial r} \mathrm{d}r + \frac{\partial f}{\partial \phi} \mathrm{d}\phi + \frac{\partial f}{\partial \theta} \mathrm{d}\theta \\
        \implies \nabla = \hat r \frac{\partial }{\partial r} + \hat \theta \frac{1}{r \sin(\theta)} & \frac{\partial}{\partial \phi} + \hat \phi \frac{1}{r} \frac{\partial}{\partial \phi}            
    \end{align*}
    %
    We need the unit vector to get further. The definition of a unit-vector is 
    %
    \begin{equation*}
        \hat n = \frac{\mathrm{d}  \pmb{r} / \mathrm{d} n}{|\mathrm{d}   \pmb{r} / \mathrm{d} n|},
    \end{equation*}
    %
    where the \(\pmb{r}\) is the position vector, given by
    %
        \begin{align*}
            \pmb{r} = x \hat x + y \hat y + z & \hat z \\
            \pmb{r} = r \cos (\phi) \sin(\theta) \hat x + r \sin (\phi) & \sin(\theta) \hat y + r \cos(\theta) \hat z
        \end{align*}
    %   
    To find the unit vectors in the spherical coordinats system, we need
    %
    \begin{align*}
        \frac{\mathrm{d} \pmb{r}}{\mathrm{d} r} = \cos (\phi) \sin(\theta) \hat x + \sin (\phi) \sin(\theta) \hat y + \cos (\theta) \hat z \\
        \bigg| \frac{\mathrm{d} \pmb{r}}{\mathrm{d} r} \bigg| = \sqrt{\cos^2(\phi) \sin^2(\theta)  + \sin (\phi) \sin^2(\theta) + \cos^2 (\theta) } = 1 \\
        \frac{\mathrm{d} \pmb{r}}{\mathrm{d} \phi} = - r \sin{(\phi)} \sin{(\theta)} \hat x + r \cos{(\phi)} \sin{(\theta)} \hat y \\
        \bigg|\frac{\mathrm{d} \pmb{r}}{\mathrm{d} \phi}\bigg| = r \sqrt{\sin^2{(\phi)} \sin^2{(\theta)} + \cos^2{(\phi)} \sin^2{(\theta)} } = r \sin{(\theta)} \\
        \frac{\mathrm{d} \pmb{r}}{\mathrm{d} \theta} = r \cos(\phi) \cos(\theta) \hat x + r \sin(\phi)\cos(\theta) \hat y - r \sin(\theta) \hat z \\
        \bigg|\frac{\mathrm{d} \pmb{r}}{\mathrm{d} \theta}\bigg| = r \sqrt{\cos^2(\phi) \cos^2(\theta) + \sin^2(\phi)\cos^2(\theta) + \sin^2(\theta)} = r
    \end{align*}
    %
    This gives us the unit vectors
    %
    \begin{align*}
        \hat r = \cos (\phi) \sin(\theta) \hat x + \sin (\phi) \sin(\theta) \hat y + \cos (\theta) \hat z \\
        \hat \phi = - \sin{(\phi)} \hat x + \cos{(\phi)} \hat y \\
        \hat \theta =  \cos(\phi) \cos(\theta) \hat x + \sin(\phi)\cos(\theta) \hat y - \sin(\theta) \hat z
    \end{align*}
    %
    To find the laplace operator, we just calculate the effect of applying \(\nabla\) twice
    %
    \begin{align*}
        \bigg( \hat r \frac{\partial }{\partial r} + \hat \theta \frac{1}{r \sin(\theta)} \frac{\partial}{\partial \phi} + \hat \phi \frac{1}{r} \frac{\partial}{\partial \theta} \bigg) & \bigg( \hat r \frac{\partial }{\partial r} + \hat \theta \frac{1}{r \sin(\theta)} \frac{\partial}{\partial \phi} + \hat \phi \frac{1}{r} \frac{\partial}{\partial \theta} \bigg) f = \\
        \bigg( \frac{\partial^2}{\partial r^2} +  \hat \theta \frac{1}{r \sin(\theta)}\frac{\partial \hat r}{\partial \phi}\frac{\partial}{\partial r} +  \frac{1}{r^2 \sin^2(\theta)} \bigg[\frac{\partial ^2}{\partial \phi^2} + & \frac{\partial \hat \theta}{\partial \phi}\frac{\partial }{\partial \phi}\bigg] + \frac{1}{r^2 \sin(\theta)}\frac{\partial \hat \phi}{\partial \phi} \frac{\partial}{\partial \phi} + \\ \hat \phi \frac{1}{r}\frac{\partial \hat r}{\partial \theta} \frac{\partial}{\partial r} + \frac{1}{r^2 \sin(\theta)} \frac{\partial \hat \theta}{\partial \theta}\frac{\partial}{\partial \phi} + &\frac{cos(\theta)}{r^2 \sin^2(\theta)}\frac{\partial}{\partial \phi} + \frac{1}{r^2}\frac{\partial \hat \phi}{\partial \theta} \bigg) f
    \end{align*}
    %
    The partial derivatives needed are
    %
    \begin{align*}
        \frac{\partial \hat \theta}{\partial \phi} = -\sin(\phi) \cos(\theta) \hat x + \cos(\phi)\cos(\theta) \hat y = \cos(\theta) \hat \phi \\ 
        \frac{\partial \hat \phi}{\partial \phi} = - \cos{(\phi)} \hat x - \sin{(\phi)} \hat y + (\sin(\theta)\hat z - \sin(\theta)\hat z c )
    \end{align*}
\end{document}